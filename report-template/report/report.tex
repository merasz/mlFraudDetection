%%%%%%%%%%%%%%%%%%%%%%%%%%%%%%%%%%%%%%%%%%%%%%%%%%%%%%%%%%%%%%%%%%%%%%%%%%%%%%%%
%2345678901234567890123456789012345678901234567890123456789012345678901234567890
%        1         2         3         4         5         6         7         8

\documentclass[a4, 10 pt, conference]{ieeeconf}  % Comment this line out if you need a4paper

%\documentclass[a4paper, 10pt, conference]{ieeeconf}      % Use this line for a4 paper

\IEEEoverridecommandlockouts                              % This command is only needed if 
                                                          % you want to use the \thanks command

\overrideIEEEmargins                                      % Needed to meet printer requirements.

% See the \addtolength command later in the file to balance the column lengths
% on the last page of the document

% The following packages can be found on http:\\www.ctan.org
%\usepackage{graphics} % for pdf, bitmapped graphics files
%\usepackage{epsfig} % for postscript graphics files
%\usepackage{mathptmx} % assumes new font selection scheme installed
%\usepackage{times} % assumes new font selection scheme installed
%\usepackage{amsmath} % assumes amsmath package installed
%\usepackage{amssymb}  % assumes amsmath package installed
\usepackage{multicol}
\usepackage{tcolorbox}
\usepackage{cuted,tcolorbox,lipsum}
\usepackage{xcolor}

\title{\LARGE \bf
Introduction to Machine Learning (SS 2021)\\ Programming Project
\vspace{-3em}
}


%\author{Someone Anyone$^{1}$ and Xiang Zhang$^{2}$% <-this % stops a space
%}


\begin{document}


\maketitle
\vspace{-3em}
\thispagestyle{empty}
\pagestyle{empty}

\begin{strip}
\begin{tcolorbox}[
size=tight,
colback=white,
boxrule=0.2mm,
left=3mm,right=3mm, top=3mm, bottom=1mm
]
{\begin{multicols}{2}

\textbf{Author 1}       \\
Last name:              \\  % Enter first name
First name:             \\  % Enter first name
Matrikel Nr.:               \\  % Enter Matrikel number

\columnbreak

\textbf{Author 2}       \\
Last name:              \\  % Enter first name
First name:             \\  % Enter first name
Matrikel Nr.:               \\  % Enter Matrikel number

\end{multicols}}
\end{tcolorbox}
\end{strip}

%%%%%%%%%%%%%%%%%%%%%%%%%%%%%%%%%%%%%%%%%%%%%%%%%%%%%%%%%%%%%%%%%%%%%%%%%%%%%%%%


{\color{blue}
\noindent The sections that your report must have are given in this template. Inside each section we provide pointers to what you should write about in that section (in blue text). 
\linebreak

\noindent \textbf{Please remove all the text in blue in your report, which should be of 2 pages (excluding references)!}
}

\section{Introduction}
\label{sec:intro}

{\color{blue}

\begin{itemize}
	\item What is the nature of your task (regression/classification)? Is it about classifying types of birds, or deciding the number of cookies an employee receives?
	\item Describe the dataset (number of features, number of instances, types of features, missing data, data imbalances, or any other relevant information).
\end{itemize}
}


\section{Implementation / ML Process}
\label{sec:methods}

{\color{blue}

\begin{itemize}
	\item Did you need to pre-process the dataset (augmenting data points/extracting features/reducing the dimensionality, etc.). Describe how you did it.
	\item Specify the method (e.g. linear regression, or SVM with these features, etc.). You don't have to describe the algorithm in detail, but rather the algorithm family and the properties of the algorithm within that family e.g. which distance functions for a decision tree, what architecture (layers and activations) for a neural network, etc. State what makes the algorithm suitable for this problem (use 2-5 lines)
	\item What are the reasons for choosing your ML method and reasons for not choosing another method?
	\item How did you choose hyperparameters (other design choices), and what are the values of the hyper parameters you chose for your final model? How did you make sure that the choice of hyperparameters works well?
\end{itemize}
}

\section{Results}
\label{sec:results}

{\color{blue}

\begin{itemize}
	\item Describe the performance of your model (in terms of the metrics for your chosen dataset) on the training and validation sets with the help of plots or/and tables.
	\item You must provide at least two separate visualizations (plot or tables) of two different things, i.e. don’t use a table and a bar plot of the metrics, that would be two of the same).
\end{itemize}
}

\section{Discussion}
\label{sec:discuss}

{\color{blue}
\begin{itemize}
	\item Analyze the results presented in the report (discuss things that contributed to the good or bad result). If it doesn’t work well, try to analyze why this is the case.
	\item Describe very briefly the things you tried, but did not choose as your final implementation (tried, but didn’t work, discarded ideas, etc.).
	\item How would you try to improve the results? What would you want to try?
\end{itemize}
}

\section{Conclusion}
\label{sec:con}

{\color{blue}

	\begin{itemize}
		\item Write a 5-10 line paragraph describing the main takeaway.
	\end{itemize}

}

%%%%%%%%%%%%%%%%%%%%%%%%%%%%%%%%%%%%%%%%%%%%%%%%%%%%%%%%%%%%%%%%%%%%%%%%%%%%%%%%



\end{document}
